% !TEX TS-program = LuaLaTeX
% !TEX encoding = UTF-8 Unicode

\chapter{Algorithme BM3D}
%-------------------------------------------General-----------------------------------
\section{Concept général}
Dans ce chapitre nous allons présenter les différentes étapes de l'algorithme BM3D sur des images de niveau de gris. Concernant l'application de l'algorithme sur des images couleurs, vous trouverez une explication dans \cite{2}. Dans la suite, nous assumons que le bruit est un bruit Gaussien et sa variance est equal à \(\sigma^2\) (\sigma \: étant la déviation standard du bruit appliqué sur l'image originale).
\newline
\newline
L'algorithme est divisé en deux étapes:
\newline
\newline
\begin{enumerate}
\item La première étape consiste à estimer le bruit par "hard thresholding" lors du filtrage collaboratif. Les paramètres de cette étapes sont représentés par l'exposant \textbf{hard}.
\item La deuxième étage est basée sur l'image bruité et sur l'estimation issue de la première étape. Cette étape utilise un filtre "Wiener Filter" lors du filtrage collaboratif. Les paramètres liés à cette étape sont représentés à l'aide de l'exposant \textbf{wiener}.
\end{enumerate}
\textit{\small{(image)}}
\newpage

%-------------------------------------------Basic estimate-----------------------------------
\section{Estimation basic (phase 1)}
On dénote par P le patch (bloc) de référence de taille \(K^{hard} * K^{hard}\).
\newline
\newline
\textit{\small{(image)}}
\subsection{Groupement ("Block matching")}
Cette première sous-étape regroupe l'ensemble des patches similaires à chaque patch de référence P. Pour chaque patch de référence P, on recherche les patches Q similaire à P dans une fenêtre centrée sur P de taille \(n^{hard} * n^{hard}\). Le groupe de patches est défini par: 
\begin{equation}
P(P) = \{Q: d(P,Q) \leq \tau^{hard}\}
\end{equation}
\newline
Ou
\begin{itemize}
\item \( \tau^{hard} \): est la distance limite pour laquelle deux patches sont considérés comme similaire.
\item \(d(P,Q) = \frac{\parallel \gamma' (P) - \gamma'(Q) \parallel_2^2}{(k^{hard})^2} \) est la distance quadratique normalisée entre deux patches\footnote{Nous avons ici calculé la distance Euclidienne carrée.}.
\item \gamma' est le seuil du filtre qui est equal à \(\lambda^{hard}_{2D}\sigma \). Pour un \sigma \leq 40, le seuil est equal à \(\lambda^{hard}_{2D}\sigma \) = 0. Pour \sigma \leq 40, tous les coefficients sont inchangés\footnote{Nous verrons au chapitre 3 (Réalisation) que nous avons omis cette étape ayant pris en compte des images avec un \sigma \leq 30 pour nos tests.}.  
\item \(\sigma^2\) \: est la variance du bruit Gaussien appliquée à l'image.
\end{itemize}

Le groupe 3D représenté par \(P \)(P) est construit en regroupant les patches similaire \(P \)(P). Pour augmenter la performance de l'algorithme, seulement les \(N^{hard} \) patches de \(P \)(P) qui sont les plus proches du patch de référence P sont gardé pour formé le groupe 3D\footnote{Au chapitre 3 nous verrons que \(N^{hard} \) doit être une puissance de 2.}. L'ordre des patches dans le groupe 3D n'a aucune influence sur le résultat. Ceux-ci peuvent être désordonnés. 
\newpage
\subsection{Filtre collaboratif}
Une fois les groupe 3D construit, le filtre collaboratif est appliqué. On applique une transformée linéaire selon Z (profondeur du groupe 3D). Cette étape est suivie par un seuillage des coefficients du domaine spectrale. Pour finir nous appliquons une transformée 3D inverse afin de connaître l'estimation pour chaque patch du groupe. 
\begin{equation}
\mathbb{P}(P)^{hard} = \tau^{hard^-1}_{3D} (\gamma (\tau^{hard}_{3D}(\mathbb{P}(P))))
\end{equation}
Ou \gamma \: est l'opérateur de seuil avec une limite \(\lambda^{hard}_{3D}\sigma\):
\[ \gamma(x) =
  \begin{cases}
    0       & \quad \text{si} \: x  \leq \lambda^{hard}_{3D}\sigma \\
    x       & \quad \text{sinon}\\
  \end{cases}
\]
Pour une raison pratique que nous démontrerons dans les prochains chapitres, la transformée 3D \(\tau^{hard}_{3D}\) du groupe 3D P(P) est effectuée en deux étapes. Nous appliquons une première transformée 2D dénoté \(\tau^{hard}_{2D}\)sur chaque patch. Ensuite une transformée 1D dénoté par \(\tau^{hard}_{1D}\) est appliquée sur la troisième dimension du groupe 3D (profondeur).  

\subsection{Agrégation}
La dernière étape est l'agrégation. Nous avons une estimation pour chaque patch et donc une estimation pour chaque pixel. Ces estimations sont sauvegardées dans deux zones tampons ainsi définis:
\[ \forall Q \in  P(P), \forall x \in Q, 
  \begin{cases}
    v(x) = v(x) + w^{hard}_P u^{hard}_{Q,P}(x) \\
    \delta(x) = \delta(x) + w^{hard}_P  \\
  \end{cases}
\]
Ou:
\begin{itemize}
\item \(v (resp. \: \delta ) \) représente le numérateur (resp. le dénominateur) de l'estimation basic de l'image obtenue à la fin de l'étape de groupage.
\item \(u^{hard}_{Q,P} \) est l'estimation du pxiel \(x \) qui appartient au patch Q obtenu lors de la phase du filtre collaboratif du patch P.
\item \( w^{hard}_P  = \begin{cases} (N^{hard}_P)^{-1}  & \quad \text{si} \: N^{hard}_P \ge 1 \\ 1  & \quad\text{sinon}\\ \end{cases} \) 
\item \( N^{hard}_P \) correspond aux nombres de coefficients (non-zero) du bloc 3D après filtrage: \(\gamma ( \tau^{hard}_{3D} (\mathbb{P}(P))) \). 
\end{itemize}
\newpage
L'intérêt de cet méthode est qu'elle donne aux patches homogènes (ou beaucoup de coefficient ont été éliminés) la priorité. Un patch avec des effets de bords sera moins pris en compte que des patches homogènes. La figure ci-dessous montre cette effet. L'algorithme donne plus d'importance aux patches vert durant la phase d'agrégation.
\newline
\newline
\textit{\small{(image)}}
\newline
\newline
Afin de réduire les effets de bords qui peuvent apparaître, nous appliquons une Kaiser window (\( k^{hard} * k^{hard}\)) après la transformée 3D inverse sur chaque patch. Cela revient à multiplier chaque coefficient de la Kaiser window avec le coefficient du patch \footnote{Nous montrerons au chapitre 4 (Résultat) que cette étape influence peu le résultat final comme mentionné dans \cite{2}.}.  
\newline 
\newline
L'estimation basic obtenue après cette première étape est donnée par:
\begin{equation}
u^{basic}(x) = \frac{\displaystyle\sum_{P}w^{hard}_P \displaystyle\sum_{Q \in P(P)}X_Q(x)u^{hard}_{Q,P}(x)}{\displaystyle\sum_{P}w^{hard}_P \displaystyle\sum_{Q \in P(P)}X_Q(x)}
\end{equation}
Avec \(X_Q(x)\) définit ainsi:
\[ X_Q(x) =
  \begin{cases}
    1       & \quad \text{si et seulement si} \: x  \in Q \\
    0       & \quad \text{sinon}\\
  \end{cases}
\]
Ce résultat est simplement obtenue en divisant les deux buffers préalablement calculés pour chaque pixel de l'image original: 
\begin{equation}
u^{basic}(x) = \frac{v(x)}{\delta(x)}
\end{equation}
\newpage

%-------------------------------------------Final estimate-----------------------------------
\section{Estimation finale (phase 2)}
Dans cette deuxième partie de l'algorithme, l'estimation basic \(u^{basic} \) issue de la première partie est accessible. Cette seconde étape applique un filtre \textit{Wiener Filter} sur l'image originale \(u \). Cette étape finale permet de restaurer plus de détails. 
\subsection{Groupement ("Block matching")}
Le regroupement des patches (calcul de la distance entre patches) est effectué sur l'estimation basic issue de la première partie de l'algorithme. Le groupe de patches similaire est définit par:
\begin{equation}
P^{basic}(P) = \{ Q: d(P,Q) \leq \tau^{wien} \}
\end{equation}
Deux 3D groupes sont formé:
 \begin{enumerate}
\item \(\mathbb{P}^{basic}(P) \) en regroupant les patches similaires  depuis l'image issue de l'estimation basic \(u^{basic}\).
\item \(\mathbb{P}(P)\) en regroupant les patches de la même manière qu'au point 1 depuis l'image bruité originale \(u\).
\end{enumerate}
Pour une question d'optimisation de l'algorithme, seul \(N^{wien} \) patches sont pris en comptes pour chacun des 3D groupes\footnote{Nous verrons que \(N^{wien} \) doit être une puissance de 2 comme \(N^{hard} \) lors de la phase 1 (estimation basic).}.
\subsection{Filtre collaboratif}
Lorsque les groupe 3D sont formés, le filtre collaboratif peut être appliqué. Pour cela, les \textit{Wiener} coefficients sont definis par:
\begin{equation}
w_P(\xi) = \frac{\mid \tau^{wien}_{3D} (\mathbb{P}^{basic}(P))(\xi) \mid^2}{\mid \tau^{wien}_{3D}(\mathbb{P}^{basic}(P))(\xi) \mid^2 + \sigma^2}
\end{equation}
\newpage
Le filtre collaboratif \textit{Wiener} de \(\mathbb{P}(P)\) est réalisé par la multiplication élément par élément de la transformée 3D de l'image bruité avec les coefficients \textit{Wiener} \(w_P\). Lors de cette étape, l'estimation d'un groupe 3D est ainsi obtenu:
\begin{equation}
\mathbb{P}^{wien}(P) = \tau_{3D}^{wien^{-1}} (w_P * \tau_{3D}^{wien}(\mathbb{P}(P)))
\end{equation}
\subsection{Agrégation}
L'agrégation est basé sur les même concepts décrit lors de la première phase de l'algorithme. Les estimations de chaque pixel seront maintenu dans deux zones tampons définis par:
\[ \forall Q \in  P(P), \forall x \in Q, 
  \begin{cases}
    v(x) = v(x) + w^{wien}_P u^{wien}_{Q,P}(x) \\
    \delta(x) = \delta(x) + w^{wien}_P  \\
  \end{cases}
\]
Ou:
\begin{itemize}
\item \(v (resp. \: \delta ) \) représente le numérateur (resp. le dénominateur) de l'estimation finale de l'image obtenue à la fin de l'étape de groupage.
\item \(u^{wien}_{Q,P} \) est l'estimation du pxiel \(x \) qui appartient au patch Q obtenu lors de la phase du filtre collaboratif du patch P.
\item \( w^{wien}_P  = \parallel w_P \parallel^{-2}_2 \). 
\end{itemize}
Comme lors de la première étape de l'algorithme, une Kaiser window \(k^{wien} * k^{wien}\) est appliqué pour réduire les effets de bords. L'estimation finale est donnée par:
\begin{equation}
u^{finale}(x) = \frac{\displaystyle\sum_{P}w^{wien}_P \displaystyle\sum_{Q \in P(P)}X_Q(x)u^{wien}_{Q,P}(x)}{\displaystyle\sum_{P}w^{wien}_P \displaystyle\sum_{Q \in P(P)}X_Q(x)}
\end{equation}
Avec \(X_Q(x)\) définit ainsi:
\[ X_Q(x) =
  \begin{cases}
    1       & \quad \text{si et seulement si} \: x  \in Q \\
    0       & \quad \text{sinon}\\
  \end{cases}
\]
L'estimation finale peut être obtenu de la même manière que l'estimation basic, en calculant (2.4) pour chaque pixel de l'image originale \(u\) \footnote{la création de deux zones tampons est approprié à l'exécution sur GPU, comme nous allons le montrer dans les prochains chapitres.}. 