% !TEX TS-program = LuaLaTeX
% !TEX encoding = UTF-8 Unicode

\chapter{Introduction}
\paragraph{}
\textit{Collaborate Filtering} est le nom de la méthode the groupage et de filtrage de l'algorithme BM3D. Cette méthode est réalisée en 4 étapes: 1) Trouver a bloc (portion de l'image) similaire à un bloc de référence et ensuite grouper les ensemble pour former un bloc 3D. 2) Appliquer une transformée 3D sur l'ensemble des bloc 3D. 3) Réduire les coefficient du monde spectrale. 4) Appliquer la transformée 3D inverse. 
\paragraph{}
En atténuant le bruit, le \textit{Collaborate Filtering} révèle les plus petits détails partagés par les groupes de blocs. Chaque bloc filtré est alors remit à sa position d'origine. Sachant que deux blocs peuvent se chevaucher, nous pouvons obtenir plusieurs estimations pour un pixel donné. C'est pourquoi, nous devons les combiner. Cette étape finale est l'\textit{Aggrégation}.
\paragraph{}
Le premier filtre collaboratif est considérablement amélioré par un second filtre \textit{Wiener Filtering}. Cette seconde étape mime la premières étapes avec deux différences. Le \textit{Block-Matching} est appliqué sur l'image filtré et non sur l'image bruité. Nous n'appliquons plus un filtre \textit{Hardthreshold} mais un filtre \textit{Wiener Filtering}. L'étape finale d'\textit{Aggrégation} reste inchangée.
\paragraph{}
L'algorithme BM3D détaillé ici est directement tiré de l'article originale \cite{1}. Plusieurs analyses préalable de l'algorithme BM3D démontre une diminution de la performance de débrouillage lorsque la déviation standard du bruit dépasse 40. Il a été démontré que pour de grande valeurs de bruit (standard deviation), les paramètres doivent être revu. Le seuil de filtrage de la première phase doit être augmenté. Il a aussi été démontré que les transformées 2D appliquées peuvent influencer l'algorithme. Dans \cite{2}, ils montrent que la meilleurs qualité d'image est atteinte en appliquant une transformée \textit{2D bi-orthogonal spline wavelet}. 
\paragraph{}
Dans ce papier nous nous sommes focaliser sur la qualité, mais plus particulièrement sur l'implémentation sur GPU (parallélisation de l'algorithme BM3D). Pour une question de simplicité, nous avons décidé d'appliquer des DCT-2D (même si nous savons que nous pouvons atteindre une meilleure qualité en appliquant d'autres transformées lors de la phase 1 et 2). Nous allons montrer dans les chapitres suivants que nous avons une bonne qualité en comparaison à \cite{3}. Pour ce qui est de la transformé 1D selon Z, nous avons opté pour la transformée d'Hadamar-Welsh \cite{4}. Nous verrons que cette méthode est efficace sur GPU. Dans \cite{1}, l'algorithme (écrit avec Matlab) est capable de traiter une image en 4 secondes. Ce temps est relativement long. Le but de ce papier est  de montrer que nous pouvons profiter de la puissance du GPU pour  réduire ce temps à quelques milisecondes, ce qui revient à montrer que BM3D peut être paraléliser.   
\paragraph{}
Dans le chapitre 2, nous présentons en détails l'algorithme BM3D. Le chapitre 3 est dédié à la réalisation sur GPU. Au chapitre 4, nous présentons nos résultats obtenus en comparaison aux travaux menés par \cite{3}. Nous présentons nos conclusions et les prochaines étapes au chapitre  5.     